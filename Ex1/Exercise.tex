%% ================================================================
%% # DBIS Databases Hand-in Template
%% 
%% Template for students to hand-in their databases exercise solutions.
%% 
%% [Databases and Information Systems Group](https://dbis.dmi.unibas.ch/)
%%
%% ## Usage
%% 
%% Fill in the required fields and write your submission
%%
%% ## Issues
%%
%% See dbisdbhandin.sty for further information.
%% ================================================================
\documentclass{article}
\usepackage{dbisdbhandin}
\usepackage[english]{babel}
%% ================================================================
%%
%% General Information
%%
%% ================================================================
%%
%% Add your information here
\course       {Databases}
\semester     {Autumn 2025}
\exerciseno   {1}
\studenta     {Aiysha Frutiger}
\studentb     {Jannick Seper}
\studentc     {Luis Tritschler}
%% Comment if you do the exercises alone


%% ================================================================
%%
%% Common Packages
%%
%% ================================================================
%%
%% Useful common packages for this course

%% Include pdfs into latex
\usepackage{pdfpages}
%% \includepdf[pages=1,pagecommand={\pagestyle{fancy}}]{filename}

%% Drawing everything, with lots of libraries
\usepackage{tikz}
%% A library providing ER prefabs
\usetikzlibrary{er}
\usetikzlibrary{positioning,arrows.meta}
%% ================================================================
%%
%% Custom Packages
%%
%% ================================================================
%%
%% Add custom packages below:
%%

% \usepackage{mypackage}

%% ================================================================
%%
%% Custom Commands
%%
%% ================================================================
%%
%% DRY: Use commands when you use something often or you'd like
%% to define it only once
%%

\newcommand*{\TikZ}{Ti\textit{k}Z}


\begin{document}
%% Required for the title
\printfront
%% ================================================================
%%
%% Description
%%
%% ================================================================

This template showcases various useful latex commands and setups.
Remove this for your actual hand-in.

%% ================================================================
%%
%% Your Solution
%%
%% ================================================================
%%
%% Add your solution below
\task{}
Relational algebra is a treat with \LaTeX, as can be seen in~\Cref{eq:ans}:

\begin{equation}
\label{eq:ans}
\pi\left[Attribute1,Attribute2\right](\sigma\left[Attribute=Something\right](Entity1) \Join Entity2)
\end{equation}

\task{}

\begin{enumerate}[(a)]
\item 
Person on Name:
\begin{equation}
\label{eq:a1}
\sigma\left[Name='Christopher Nolan'\right](Person)
\end{equation}

Activity on SceneAuthor:
\begin{equation}
\label{eq:a2}
\sigma\left[Activity='director'\right](SceneAuthor)
\end{equation}

Add join on both:
\begin{equation}
\label{eq:a-compact}
\pi\left[MovieID,SceneID\right](\sigma\left[Name='Christopher Nolan' \land Activity='director'\right]
  (Person \Join SceneAuthor))
\end{equation}


\end{enumerate}

\end{document}
