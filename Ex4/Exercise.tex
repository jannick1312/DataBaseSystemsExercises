%% ================================================================
%% # DBIS Databases Hand-in Template
%% 
%% Template for students to hand-in their databases exercise solutions.
%% 
%% [Databases and Information Systems Group](https://dbis.dmi.unibas.ch/)
%%
%% ## Usage
%% 
%% Fill in the required fields and write your submission
%%
%% ## Issues
%%
%% See dbisdbhandin.sty for further information.
%% ================================================================
\documentclass{article}
\usepackage{dbisdbhandin}
\usepackage[english]{babel}
%% ================================================================
%%
%% General Information
%%
%% ================================================================
%%
%% Add your information here
\course       {Databases}
\semester     {Autumn 2025}
\exerciseno   {4}
\studenta     {Aiysha Frutiger}
\studentb     {Jannick Seper}
\studentc     {Luis Tritschler}
%% Comment if you do the exercises alone


%% ================================================================
%%
%% Common Packages
%%
%% ================================================================
%%
%% Useful common packages for this course

%% Include pdfs into latex
\usepackage{pdfpages}
%% \includepdf[pages=1,pagecommand={\pagestyle{fancy}}]{filename}

%% Drawing everything, with lots of libraries
\usepackage{tikz}
%% A library providing ER prefabs
\usetikzlibrary{er}
\usetikzlibrary{positioning,arrows.meta}
%% ================================================================
%%
%% Custom Packages
%%
%% ================================================================
%%
%% Add custom packages below:
%%

% \usepackage{mypackage}

%% ================================================================
%%
%% Custom Commands
%%
%% ================================================================
%%
%% DRY: Use commands when you use something often or you'd like
%% to define it only once
%%

\newcommand*{\TikZ}{Ti\textit{k}Z}


\begin{document}

\printfront

\task{}
\begin{enumerate}[a)]
    \item
        \lstinputlisting[language=SQL,frame=single,numbers=left,numberstyle=\tiny]{task1_a.sql}

    \begin{samepage}
    \item
        \lstinputlisting[language=SQL,frame=single,numbers=left,numberstyle=\tiny]{task1_b.sql}
    \end{samepage}
    \newpage

    \item
        \lstinputlisting[language=SQL,frame=single,numbers=left,numberstyle=\tiny]{task1_c.sql}
\end{enumerate}



\newpage
\task{}

a)

We first determine how many tuples fit on one data page and then how many pages
are required per relation.
The page size is $4096\,\mathrm{B}$, of which $96\,\mathrm{B}$ are reserved for the page header.
Thus, the usable space per page is
\[
4096\,\mathrm{B} - 96\,\mathrm{B} = 4000\,\mathrm{B}.
\]
Since data pages are filled only up to $90\%$, the effective capacity for tuples is
\[
0.9 \cdot 4000\,\mathrm{B} = 3600\,\mathrm{B}.
\]
Each tuple additionally requires one slot array pointer of $6\,\mathrm{B}$. Hence, the
effective record size is
\[
\text{record size} = \text{tuple size} + 6\,\mathrm{B}.
\]
The number of tuples per page is then
\[
\text{tuples per page} = \left\lfloor \frac{3600}{\text{record size}} \right\rfloor,
\]
and the number of pages is
\[
\text{pages} = \left\lceil \frac{\#\text{tuples}}{\text{tuples per page}} \right\rceil.
\]
\paragraph{Movie}
Average tuple size: $100\,\mathrm{B}$, record size: $100 + 6 = 106\,\mathrm{B}$.
\[
\text{tuples per page} = \left\lfloor \frac{3600}{106} \right\rfloor = 33,
\qquad
\text{pages} = \left\lceil \frac{1000}{33} \right\rceil = 31.
\]

\paragraph{Scene}
Average tuple size: $50\,\mathrm{B}$, record size: $50 + 6 = 56\,\mathrm{B}$.
\[
\text{tuples per page} = \left\lfloor \frac{3600}{56} \right\rfloor = 64,
\qquad
\text{pages} = \left\lceil \frac{10000}{64} \right\rceil = 157.
\]

\paragraph{Person}
Average tuple size: $40\,\mathrm{B}$, record size: $40 + 6 = 46\,\mathrm{B}$.
\[
\text{tuples per page} = \left\lfloor \frac{3600}{46} \right\rfloor = 78,
\qquad
\text{pages} = \left\lceil \frac{6000}{78} \right\rceil = 77.
\]

\paragraph{Actor}
Same tuple size as \texttt{Person}, hence same record size:
$40 + 6 = 46\,\mathrm{B}$.
\[
\text{tuples per page} = \left\lfloor \frac{3600}{46} \right\rfloor = 78,
\qquad
\text{pages} = \left\lceil \frac{30000}{78} \right\rceil = 385.
\]

\begin{center}
\begin{tabular}{|l||c|c|}
\hline
Relation & Tuples per page & Number of pages \\
\hline\hline
\texttt{Movie}  & 33 & 31  \\ \hline
\texttt{Scene}  & 64 & 157 \\ \hline
\texttt{Person} & 78 & 77  \\ \hline
\texttt{Actor}  & 78 & 385 \\ \hline
\end{tabular}
\end{center}


b) 

The page size is $4096\,\mathrm{B}$ with a page header of $96\,\mathrm{B}$, thus
\[
4096\,\mathrm{B} - 96\,\mathrm{B} = 4000\,\mathrm{B}
\]
are available per page. Index pages are filled up to $70\%$, hence the effective
capacity is
\[
C = 0.7 \cdot 4000\,\mathrm{B} = 2800\,\mathrm{B}.
\]
Pointers in the B$^+$-tree and tuple identifiers (TIDs) have a length of $6\,\mathrm{B}$,
data type \texttt{NUMBER} has $10\,\mathrm{B}$, and the average \texttt{MovieTitle}
length is $90\,\mathrm{B}$.

\paragraph{MovieIDIdx}

Key size $k = 10\,\mathrm{B}$.  
In a leaf node, one entry consists of $(\text{key}, \text{TID})$ with size
$10 + 1 \cdot 6 = 16\,\mathrm{B}$. We reserve $2$ pointers of $6\,\mathrm{B}$ on each
leaf page (e.g.\ sibling pointers):
\[
t_{\text{leaf}}
 = \left\lfloor
   \frac{(4096 - 96 - (2 \cdot 6)) \cdot 0.7}{10 + (1 \cdot 6)}
   \right\rfloor
 = 174.
\]
For inner nodes, one entry consists of $(\text{key}, \text{child pointer})$ of size
$10 + 6 = 16\,\mathrm{B}$, plus an additional pointer of $6\,\mathrm{B}$:
\[
e_i
 = \left\lfloor
   \frac{(4096 - 96) \cdot 0.7 - 6}{10 + 6}
   \right\rfloor
 = 174.
\]
With $1000$ movies, the number of leaf nodes is
\[
n_{\text{leaf}}
 = \left\lceil \frac{1000}{174} \right\rceil
 = 6.
\]
An inner node can point to $e_i + 1 = 175$ children, thus a single root node suffices:
\[
n_{\text{inner}} = \left\lceil \frac{6}{175} \right\rceil = 1.
\]
The total index size is
\[
i_{\text{size}} = (6 + 1) \cdot 4 = 28\,\text{KiB}.
\]

\paragraph{MovieTitleIdx}

Key size $k = 90\,\mathrm{B}$.  
Leaf entry size: $90 + 1 \cdot 6 = 96\,\mathrm{B}$:
\[
t_{\text{leaf}}
 = \left\lfloor
   \frac{(4096 - 96 - (2 \cdot 6)) \cdot 0.7}{90 + (1 \cdot 6)}
   \right\rfloor
 = 29.
\]
Inner node:
\[
e_i
 = \left\lfloor
   \frac{(4096 - 96) \cdot 0.7 - 6}{90 + 6}
   \right\rfloor
 = 29.
\]
With $1000$ movies:
\[
n_{\text{leaf}}
 = \left\lceil \frac{1000}{29} \right\rceil
 = 35.
\]
Each inner node can have up to $e_i + 1 = 30$ children, so the number of inner nodes
one level above the leaves is
\[
n_{\text{inner1}} = \left\lceil \frac{35}{30} \right\rceil = 2,
\]
and with one root node above:
\[
n_{\text{inner2}} = \left\lceil \frac{2}{30} \right\rceil = 1,
\qquad
n_{\text{inner}} = 2 + 1 = 3.
\]
Thus, the total index size is
\[
i_{\text{size}} = (35 + 3) \cdot 4 = 152\,\text{KiB}.
\]

\paragraph{ActorIdx}

There are $30000$ tuples in \texttt{Actor} and $6000$ tuples in \texttt{Person},
so on average there are
\[
r_k = \frac{30000}{6000} = 5
\]
\texttt{Actor} entries per \texttt{PID}.  
In the leaf nodes of \texttt{ActorIdx}, we store one entry per distinct key \texttt{PID},
together with a list of $r_k$ TIDs. Hence, the average leaf entry size is
\[
10 + r_k \cdot 6 = 10 + 5 \cdot 6 = 40\,\mathrm{B}.
\]
We again reserve $2$ pointers of $6\,\mathrm{B}$ per leaf page:
\[
t_{\text{leaf}}
 = \left\lfloor
   \frac{(4096 - 96 - (2 \cdot 6)) \cdot 0.7}{10 + (5 \cdot 6)}
   \right\rfloor
 = 69.
\]
For inner nodes, we only store $(\text{key}, \text{child pointer})$:
\[
e_i
 = \left\lfloor
   \frac{(4096 - 96) \cdot 0.7 - 6}{10 + 6}
   \right\rfloor
 = 174.
\]
We have $6000$ distinct keys (PIDs), hence
\[
n_{\text{leaf}}
 = \left\lceil \frac{6000}{69} \right\rceil
 = 87.
\]
An inner node can point to up to $e_i + 1 = 175$ children, so
\[
n_{\text{inner}}
 = \left\lceil \frac{87}{175} \right\rceil
 = 1.
\]
The total index size is therefore
\[
i_{\text{size}} = (87 + 1) \cdot 4 = 352\,\text{KiB}.
\]

\begin{center}
\begin{tabular}{|l||c|c||c|c||c|}
\hline
 & \multicolumn{2}{c||}{Entries per} & \multicolumn{2}{c||}{Number of} &
 \multicolumn{1}{c|}{Index Size} \\
Index & leaf nodes & inner nodes & leaf nodes & inner nodes & [KiB] \\
\hline\hline
\texttt{MovieIDIdx}     & 174 & 174 & 6   & 1 & 28  \\ \hline
\texttt{MovieTitleIdx}  & 29  & 29  & 35  & 3 & 152 \\ \hline
\texttt{ActorIdx}       & 69  & 174 & 87  & 1 & 352 \\ \hline
\end{tabular}
\end{center}

\newpage

c)

\[
\texttt{SELECT * FROM Movie WHERE MovieID = 4711;}
\]
\[
\texttt{SELECT * FROM Movie WHERE MovieTitle = 'Opelgang';}
\]
\[
\texttt{SELECT * FROM Actor WHERE PID = 1199;}
\]

\paragraph{Movie.MovieID}

For \texttt{MovieIDIdx} we have $n_{\text{leaf}} = 6$ leaf nodes and a maximum
fanout of $175$ children per inner node. Hence, the height of the B$^+$-tree
(including root and leaves) is
\[
h = \left\lceil \log_{175} 6 \right\rceil + 1 = 2 .
\]
Thus we access $h = 2$ index pages plus one data page in \texttt{Movie}:
\[
\text{total page accesses} = 2 + 1 = 3.
\]

\paragraph{Movie.MovieTitle}

For \texttt{MovieTitleIdx} we have $n_{\text{leaf}} = 35$ leaf nodes and a fanout
of $30$ children per inner node. The height is
\[
h = \left\lceil \log_{30} 35 \right\rceil + 1 = 3 .
\]
Again, we access one data page in \texttt{Movie}:
\[
\text{total page accesses} = 3 + 1 = 4.
\]

\paragraph{Actor.PID}

For \texttt{ActorIdx} we have $n_{\text{leaf}} = 87$ leaf nodes and a fanout of
$175$ children per inner node. The height is
\[
h = \left\lceil \log_{175} 87 \right\rceil + 1 = 2 .
\]
On average there are $r_k = 5$ \texttt{Actor} tuples per \texttt{PID}, so we expect
to access $5$ data pages in \texttt{Actor}:
\[
\text{total page accesses} = 2 + 5 = 7.
\]

\begin{center}
\begin{tabular}{|l|l||c|}
\hline
Relation & Attribute & Page accesses \\
\hline\hline
\texttt{Movie} & \texttt{MovieID}    & 3 \\ \hline
\texttt{Movie} & \texttt{MovieTitle} & 4 \\ \hline
\texttt{Actor} & \texttt{PID}        & 7 \\ \hline
\end{tabular}
\end{center}

\end{document}
