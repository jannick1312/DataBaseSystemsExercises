\task{Relational Algebra and Operator Tree}

We are given the relations
\[
U(a,b,c),\quad V(c,d,e),\quad W(e,f,g)
\]
and the following SQL query:
\begin{verbatim}
SELECT b, f
FROM (
  SELECT V.c, V.d, W.e, W.f, W.g
  FROM V, W
  WHERE V.e = W.e
    AND f = 2
    AND g = 10
) 
JOIN U ON U.c = V.c
WHERE a LIKE 'Ben';
\end{verbatim}

\subsection{Relational algebra expression}

A canonical translation of the query into relational algebra is:
\[
\pi_{b,f}\Bigl(
  \sigma_{a \,\text{LIKE}\, 'Ben'}\bigl(
    \bigl(
      \pi_{V.c,\,V.d,\,W.e,\,W.f,\,W.g}\bigl(
        \sigma_{V.e = W.e \,\land\, W.f = 2 \,\land\, W.g = 10}
        (\,V \times W\,)
      \bigr)
    \bigr)
    \Join_{U.c = V.c}
    U
  \bigr)
\Bigr).
\]

\subsection{Algebraic operator tree}

The corresponding algebraic operator tree is shown in
Figure~\ref{fig:op-tree}.

\begin{figure}[h]
  \centering
  \begin{tikzpicture}[nodes={draw, circle}, node distance=4em]
    % Root: projection π_{b,f}
    \node (pi1) {$\pi$};
    \node[right=0.5em of pi1, draw=none] {$b, f$};

    % Selection σ_{a LIKE 'Ben'}
    \node[below of=pi1] (sigma1) {$\sigma$} edge (pi1);
    \node[right=0.5em of sigma1, draw=none] {$a \text{ LIKE 'Ben'}$};

    % Join ⨝_{U.c = V.c}
    \node[below of=sigma1] (join1) {$\Join$} edge (sigma1);
    \node[right=0.5em of join1, draw=none] {$U.c = V.c$};

    % Left child of join1: relation U
    \node[below left of=join1] (U) {U} edge (join1);

    % Right child of join1: subtree (inner SELECT)
    \node[below right of=join1] (pi2) {$\pi$} edge (join1);
    \node[right=0.5em of pi2, draw=none] {$V.c, V.d, W.e, W.f, W.g$};

    % σ_{V.e = W.e ∧ W.f = 2 ∧ W.g = 10}
    \node[below of=pi2] (sigma2) {$\sigma$} edge (pi2);
    \node[right=0.5em of sigma2, draw=none] {$V.e = W.e \land W.f = 2 \land W.g = 10$};

    % × with children V and W
    \node[below of=sigma2] (prod) {$\times$} edge (sigma2);
    \node[below left of=prod] (V) {V} edge (prod);
    \node[below right of=prod] (W) {W} edge (prod);
  \end{tikzpicture}
  \caption{Algebraic operator tree for the given SQL query.}
  \label{fig:op-tree}
\end{figure}

