\setcounter{subtask}{0}
\subtask{Page accesses for complex query}

We consider
\[
Q:\ \texttt{SELECT COUNT(*) FROM R WHERE A = 321 AND B <> 50;}
\]

We approximate the combined filter factor assuming independence and
uniformity for $A$ and $B$:
\[
FF(A = 321 \land B \neq 50)
= FF(A = 321)\cdot FF(B \neq 50)
= \frac{1}{1000}\cdot\Bigl(1 - \frac{1}{1000}\Bigr)
\approx 0.000999.
\]
With $\Card(R) = 50\,000$ this gives an expected
\[
\Card(R \mid A = 321 \land B \neq 50)
\approx 50\,000 \cdot 0.000999 \approx 50
\]
matching tuples.

\begin{enumerate}[i]

  \item \textbf{R (no index)}\\
  No index on $A$ or $B$ $\Rightarrow$ full table scan:
  \[
  C_Q(\texttt{R}) = NPages(R) = \mathbf{123}.
  \]

  \item \textbf{RA (unclustered index on A)}\\
  Use index on $A$ for $A = 321$.  Index search cost:
  $C_{IS} = h = 2$ (height from Task~2).  
  Unclustered index $\Rightarrow$ in the worst case one data page per
  qualifying tuple:
  \[
  C_Q(\texttt{RA})
  = C_{IS} + \#\text{data pages}
  \approx 2 + 50
  = \mathbf{52}.
  \]

  \item \textbf{RB (unclustered index on B)}\\
  Condition $B \neq 50$ is almost unselective (matches nearly all
  tuples), so using the index would touch almost all data pages.
  Cheaper is a full table scan:
  \[
  C_Q(\texttt{RB}) \approx NPages(R) = \mathbf{123}.
  \]

  \item \textbf{RAB (indexes on A and B)}\\
  Index on $A$ is useful (selective), index on $B$ is not (due to
  $B \neq 50$).  We use the same strategy as for \texttt{RA}:
  \[
  C_Q(\texttt{RAB}) \approx C_Q(\texttt{RA}) = \mathbf{52}.
  \]

  \item \textbf{RA\$ (clustered, direct index on A)}\\
  Tuples with the same value of $A$ are stored contiguously.  The
  number of tuples with $A = 321$ and $B \neq 50$ is still about $50$,
  but all of them fit into \emph{one} data page (each page holds $407$
  tuples). Hence:
  \[
  C_Q(\texttt{RA\$})
  \approx C_{IS} + \#\text{data pages}
  \approx 2 + 1
  = \mathbf{3}.
  \]

  \item \textbf{RC (combined index on (A,B))}\\
  Combined key $(A,B)$ allows us to jump directly to the range
  $A=321$ in the index and then scan only the leaf entries for that
  value, filtering by $B \neq 50$.  As in Task~2, the index cost for a
  query on $A$ is
  \[
  C(A = 321) = (h - 1) + \bigl\lceil FF(A=321)\cdot n_{\text{leaf}} \bigr\rceil
  = 5,
  \]
  and the additional $B \neq 50$ condition is applied while scanning
  those leaf pages.  We thus approximate:
  \[
  C_Q(\texttt{RC}) \approx \mathbf{5}.
  \]

\end{enumerate}

\begin{center}
\begin{tabular}{|l||c|}
\hline
Layout & \# page accesses for $Q$ \\ \hline\hline
\texttt{R}    & 123 \\ \hline
\texttt{RA}   & 52  \\ \hline
\texttt{RB}   & 123 \\ \hline
\texttt{RAB}  & 52  \\ \hline
\texttt{RA\$} & 3   \\ \hline
\texttt{RC}   & 5   \\ \hline
\end{tabular}
\end{center}