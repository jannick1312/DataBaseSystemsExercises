%% ================================================================
%% # DBIS Databases Hand-in Template
%% 
%% Template for students to hand-in their databases exercise solutions.
%% 
%% [Databases and Information Systems Group](https://dbis.dmi.unibas.ch/)
%%
%% ## Usage
%% 
%% Fill in the required fields and write your submission
%%
%% ## Issues
%%
%% See dbisdbhandin.sty for further information.
%% ================================================================
\documentclass{article}
\usepackage{dbisdbhandin}
\usepackage[english]{babel}
%% ================================================================
%%
%% General Information
%%
%% ================================================================
%%
%% Add your information here
\course       {Databases}
\semester     {Autumn 2025}
\exerciseno   {1}
\studenta     {Aiysha Frutiger}
\studentb     {Jannick Seper}
\studentc     {Luis Tritschler}
%% Comment if you do the exercises alone


%% ================================================================
%%
%% Common Packages
%%
%% ================================================================
%%
%% Useful common packages for this course

%% Include pdfs into latex
\usepackage{pdfpages}
%% \includepdf[pages=1,pagecommand={\pagestyle{fancy}}]{filename}

%% Drawing everything, with lots of libraries
\usepackage{tikz}
%% A library providing ER prefabs
\usetikzlibrary{er}
\usetikzlibrary{positioning,arrows.meta}
%% ================================================================
%%
%% Custom Packages
%%
%% ================================================================
%%
%% Add custom packages below:
%%

% \usepackage{mypackage}

%% ================================================================
%%
%% Custom Commands
%%
%% ================================================================
%%
%% DRY: Use commands when you use something often or you'd like
%% to define it only once
%%

\newcommand*{\TikZ}{Ti\textit{k}Z}


\begin{document}

\printfront

\task{} %Task 1
\begin{enumerate}[a)]
    \item \texttt{AHV $\rightarrow$ (FullName, Birthday, Zip, country)} \newline
        \texttt{FullName $\rightarrow$ (FirstName, LastName)} \newline
        \texttt{Birthday $\rightarrow$ YearOfBirth} \newline
        \texttt{(County, Zip) $\rightarrow$ City}

    The \texttt{AHV} uniquely identifies each resident. From these attributes, all others can be derived transitively:
    \texttt{FullName} determines \texttt{(FirstName, LastName)}, \texttt{Birthday} determines \texttt{YearOfBirth}, and \texttt{(Country, Zip)} determines \texttt{City}.
    We thought about including \texttt{(Country, City) $\rightarrow$ Zip} but there are cities with multiple zips (Zürich for example) and that's why we did not include this dependency.


    \item With this functional dependencies we compute the attribute closure for \texttt{AHV}. \texttt{F$^+$} is \texttt{(AHV, FullName, Birthday, Zip, Country, FirstName, LastName, YearOfBirth, City) \newline = sch(Resident)}. Since no subset of \texttt{AHV} determines all attributes, \texttt{AHV} is minimal and therefore the only candidate key.
    

    \item The relation Resident is in 2NF, since the only candidate key is AHV and therefore no partial dependencies on a subset of a composite key can exist. It is not in Third Normal Form, because there are several transitive dependencies. For example: \texttt{AHV $\rightarrow$ (Zip, country)} and \texttt{(County, Zip) $\rightarrow$ City}, hence City is transitively dependent on AHV.
    Version in 3NF: \newline
    \texttt{Resident(\uline{AHV}, \uwave{FullName}, \uwave{Birthday}, \uwave{Zip, Country})} \newline
    \texttt{Name(\uline{FullName}, FirstName, LastName)} \newline
    \texttt{Birthday(\uline{Birthday}, YearOfBirth)} \newline
    \texttt{Location(\uline{Zip, Country}, City)} 
\end{enumerate}

\task{}  %Task 2
\begin{enumerate}[a)]
    \item \texttt{\{\{A, B, C\}, \{A, C, D\}\}}
    \item 
    \begin{enumerate}[i]
        \item Because for FD2 \texttt{B} on the left is not a superkey, R is not in BCNF.
        \item Because for FD3 \texttt{E} on the right which is not part of a primary key, R is not in 3NF.
        \item Because in FD3, \texttt{E} (non-prime attribute) is dependent on \texttt{C,D} which is a subset of a candidate key, R is not in 2NF \newline
        
        R is only in 1NF.
    \end{enumerate}
\end{enumerate}

\task{}  %Task 3
\begin{enumerate}[a)]
    \item \texttt{\{\{A, C\}, \{A, B\}\}}
    \item 
    \begin{enumerate}[i]
        \item S is in 2NF because in FD1, neither \texttt{A} nor \texttt{B} alone determine \texttt{D} and \texttt{C} is in the candidate keys. Also in FD2, although \texttt{C} is a proper subset of a candidate key, B is a primary key and therefore all rules for 2NF hold true.
        \item S is also in 2NF because in FD1, \texttt{{A,B}} is a candidate key and therefore a superkey and therefore 2NF is satisfied. Because \texttt{C} alone is not a superkey, but \texttt{B} is a prime key, this FD also satisfies 3NF.
        \item However, because \texttt{C} alone is not a superkey, it does not satisfie FD2 and therefore is not a BCNF.
    \end{enumerate}
\end{enumerate}

\task{} %Task 4

\begin{enumerate}[a)]
  \item Candidate keys:\\
        $A^+ = {A,B,C,D,U}$ not all attributes, so not candidate key alone.\\
        $(A,E)^+ = {A,B,C,D,E,U,V,W}$ all attributes, so candidate key {A,E}.\\
        $(A,W)^+ = {A,B,C,D,E,U,V,W}$ all attributes, so candidate key {A,W}.\\
        because W and E also not candidate key alone, $\{A,E\}$ and $\{A,W\}$ both minimal candidate keys.
        Because A only on left side, A needs to be part of all candidate keys.
  \item All possible functional dependencies:\\
      \begin{itemize}
        \item $A \to B$
        \item $A \to C$
        \item $A \to D$
        \item $A \to U$
        \item $A \to \{B,C,D,U\}$ (Union)
        \item $A,E \to V,W$ (Pseudotransistivity from $A \to D$ and $D,E \to V,W$)
        \item $A,E \to V$
        \item $A,E \to W$
        \item ($D,W \to V,W$ (Pseudotransistivity from $W \to E$ and $D,E \to V,W$))
        \item $A,W \to W,V$ (Pseudotransistivity from $A \to D$ and $D,W \to V,W$)
        \item $A,W \to W$
        \item $A,W \to V$
        \item $A,W \to E$ (Transistivity from $A,W \to W$ and $W \to E$)
      \end{itemize}

\end{enumerate}

\task{} %Task 5
\begin{enumerate}[a)]
    %.........5a...........
    \item \textbf{Task 5(a) – Candidate Keys of Relation \texttt{U}}
    \textbf{Given:}  
    Functional dependencies:
    \[
    \{\, \texttt{CU} \to \texttt{D},\; \texttt{D} \to \texttt{E,V,W,U,A},\; \texttt{U} \to \texttt{B},\; \texttt{V} \to \texttt{A} \,\}.
    \]

    \textbf{Identify attributes that must be in every key}  
    No functional dependency determines \texttt{C},  
    therefore \texttt{C} must be included in every candidate key.

    \textbf{Determine attribute closures}  
    \begin{itemize}
      \item For \texttt{\{C,U\}}:  
        From \texttt{CU → D} we obtain \texttt{D}.  
        Then, \texttt{D → E,V,W,U,A} and \texttt{U → B}.  
        Hence,
        \[
        (\texttt{CU})^+ = \{\texttt{A,B,C,D,E,U,V,W}\}.
        \]
        $\Rightarrow$ \texttt{\{C,U\}} is a superkey.  
        Since neither \texttt{C}$^+$ nor \texttt{U}$^+$ contains all attributes,  
        \texttt{\{C,U\}} is \textbf{minimal} and therefore a \textbf{candidate key}.

      \item For \texttt{\{C,D\}}:  
        From \texttt{D → U,A,E,V,W} and \texttt{U → B} we obtain
        \[
        (\texttt{CD})^+ = \{\texttt{A,B,C,D,E,U,V,W}\}.
        \]
        $\Rightarrow$ \texttt{\{C,D\}} is a superkey.  
        Since neither \texttt{C} nor \texttt{D} alone determines all attributes,  
        it is also \textbf{minimal} and thus a \textbf{candidate key}.
    \end{itemize}

    \textbf{Check other combinations}  
    Other sets such as \texttt{\{C,V\}} or \texttt{\{C,A\}} do not determine \texttt{U} or \texttt{D},  
    and therefore cannot be superkeys.  
    Supersets of \texttt{\{C,U\}} or \texttt{\{C,D\}} are not minimal.

    \textbf{Result:}
    \[
    \boxed{\text{The candidate keys of } \texttt{U} \text{ are } \texttt{\{C,U\}} \text{ and } \texttt{\{C,D\}}.}
    \]


  %.........5b...........
\item 
\textbf{Given:}  
Relation \texttt{U(A,B,C,D,E,U,V,W)} with functional dependencies  
\[
\{\, \texttt{CU} \to \texttt{D},\; \texttt{D} \to \texttt{E,V,W,U,A},\; \texttt{U} \to \texttt{B},\; \texttt{V} \to \texttt{A} \,\}.
\]

\textbf{Check 1NF}  
All attribute domains are atomic.  
\[
\Rightarrow \texttt{U} \text{ is in 1NF.}
\]

\textbf{Check 2NF}  
A relation is in 2NF if every non-prime attribute is fully functionally dependent on each candidate key.  
Candidate keys are \texttt{\{C,U\}} and \texttt{\{C,D\}}.

\begin{itemize}
  \item \texttt{U → B}: \texttt{U} is only part of key \texttt{\{C,U\}} $\Rightarrow$ partial dependency.
  \item \texttt{D → E,V,W}: \texttt{D} is only part of key \texttt{\{C,D\}} $\Rightarrow$ partial dependency.
\end{itemize}

$\Rightarrow$ \textit{Violates 2NF.}

\textbf{Check 3NF}  
A relation is in 3NF if, for every FD \texttt{X → A}, either \texttt{X} is a superkey or \texttt{A} is a key attribute.  
Since \texttt{D → E,V,W} and \texttt{U → B} hold, and neither \texttt{D} nor \texttt{U} are superkeys,  
these dependencies violate 3NF.

\[
\Rightarrow \texttt{U} \text{ is not in 3NF.}
\]

\textbf{Check BCNF}  
In BCNF, every determinant must be a superkey.  
Dependencies \texttt{D → E,V,W}, \texttt{U → B}, and \texttt{V → A} violate this rule.  
\[
\Rightarrow \texttt{U} \text{ is not in BCNF.}
\]

\textbf{Result:}  
Since \texttt{U} violates 2NF (and therefore 3NF and BCNF), it is only in:
\[
\boxed{\text{First Normal Form (1NF)}.}
\]


  %.........5c...........
\item 
\textbf{Minimal Cover (Canonical Form)}  

Given functional dependencies (already minimal on both sides):
\[
\{\, \texttt{CU} \to \texttt{D},\;
    \texttt{D} \to \texttt{E,V,W,U,A},\;
    \texttt{U} \to \texttt{B},\;
    \texttt{V} \to \texttt{A} \,\}.
\]
No attribute on the left–hand side can be removed (neither \texttt{C} nor \texttt{U} in \texttt{CU→D}),  
and none of the dependencies is implied by the others.  
Hence, this is already the \textbf{minimal cover}.

\medskip
\textbf{3NF Synthesis}

For each determinant, create one relation that includes all attributes from its FD group.  
Since one of them (\texttt{R\textsubscript{CU}}) contains the candidate key \texttt{\{C,U\}},  
the resulting decomposition will be lossless and dependency-preserving.

\[
\begin{aligned}
\texttt{R\textsubscript{CU}(C,U,D)} &\quad\text{from }\texttt{CU→D}\\
\texttt{R\textsubscript{D}(D,U,A,E,V,W)} &\quad\text{from }\texttt{D→U,A,E,V,W}\\
\texttt{R\textsubscript{U}(U,B)} &\quad\text{from }\texttt{U→B}\\
\texttt{R\textsubscript{V}(V,A)} &\quad\text{from }\texttt{V→A}
\end{aligned}
\]

Since \texttt{R\textsubscript{CU}} includes the key \texttt{\{C,U\}},  
the natural join of all projections is \textbf{lossless},  
and all dependencies are \textbf{preserved} (property of the 3NF-synthesis algorithm).

\medskip
\textbf{Normal Forms of the Resulting Relations}

\begin{itemize}
  \item \texttt{R\textsubscript{CU}(C,U,D)}  
        – Contains FDs \texttt{CU→D} and \texttt{D→U}.  
        – Satisfies 3NF (each non-trivial FD has either a key or a key attribute on the RHS).  
        – Not in BCNF because \texttt{D→U} holds while \texttt{D} is not a key.

  \item \texttt{R\textsubscript{D}(D,U,A,E,V,W)}  
        – \texttt{D} is a key for all its dependencies → BCNF.  
        – However, \texttt{V→A} also holds globally and both attributes occur here,  
          so to maintain BCNF we separate \texttt{R\textsubscript{V}(V,A)}.

  \item \texttt{R\textsubscript{U}(U,B)} – \texttt{U} is key → BCNF.  
  \item \texttt{R\textsubscript{V}(V,A)} – \texttt{V} is key → BCNF.
\end{itemize}

\medskip
\textbf{Variant B (cleaner BCNF form):}  
Split \texttt{R\textsubscript{D}} into individual relations for each RHS attribute:
\[
\texttt{R\textsubscript{DU}(D,U)},\;
\texttt{R\textsubscript{DA}(D,A)},\;
\texttt{R\textsubscript{DE}(D,E)},\;
\texttt{R\textsubscript{DV}(D,V)},\;
\texttt{R\textsubscript{DW}(D,W)}.
\]
Each of these is in BCNF because \texttt{D} is the key in its relation.  
\texttt{R\textsubscript{CU}(C,U,D)} remains in 3NF (not BCNF).  
\texttt{R\textsubscript{U}(U,B)} and \texttt{R\textsubscript{V}(V,A)} are BCNF.

\medskip
\textbf{Final Normal Forms}

\[
\begin{aligned}
\texttt{R\textsubscript{CU}(C,U,D)} &\rightarrow \text{3NF (not BCNF)}\\
\texttt{R\textsubscript{DU}(D,U)},\;
\texttt{R\textsubscript{DA}(D,A)},\;
\texttt{R\textsubscript{DE}(D,E)},\;
\texttt{R\textsubscript{DV}(D,V)},\;
\texttt{R\textsubscript{DW}(D,W)} &\rightarrow \text{BCNF}\\
\texttt{R\textsubscript{U}(U,B)},\;
\texttt{R\textsubscript{V}(V,A)} &\rightarrow \text{BCNF}
\end{aligned}
\]

\textbf{Result:}  
The decomposition is \textbf{lossless} (a key relation is retained)  
and \textbf{dependency-preserving}.  
All relations are in BCNF except \texttt{R\textsubscript{CU}},  
which remains in 3NF to ensure dependency preservation.
\[
\boxed{\text{Highest possible normal form with dependency preservation: 3NF.}}
\]

 %.........5d...........
 \item 
\textbf{Goal:} Find a lossless BCNF decomposition of \texttt{U} and compare it to the 3NF decomposition from (c).

\textbf{Observation.} In the 3NF design of (c), the only non-BCNF relation is \texttt{R\textsubscript{CU}(C,U,D)} due to the dependency \texttt{D → U} with \texttt{D} not being a key of \texttt{R\textsubscript{CU}}.

\medskip
\textbf{BCNF Step.} Decompose \texttt{R\textsubscript{CU}(C,U,D)} using \texttt{D → U}:
\[
\texttt{R\textsubscript{CU}(C,U,D)} \;\Rightarrow\; 
\texttt{R\textsubscript{CD}(C,D)} \quad \text{and} \quad \texttt{R\textsubscript{DU}(D,U)}.
\]
This split is \textbf{lossless} because the common attribute \texttt{D} is a key in \texttt{R\textsubscript{DU}}.

\medskip
\textbf{Final BCNF schema.} Starting from the 3NF set in (c) and replacing \texttt{R\textsubscript{CU}} by its BCNF split, we obtain:
\[
\begin{aligned}
&\texttt{R\textsubscript{CD}(C,D)},\;
\texttt{R\textsubscript{DU}(D,U)},\;
\texttt{R\textsubscript{DA}(D,A)},\;
\texttt{R\textsubscript{DE}(D,E)},\;
\texttt{R\textsubscript{DV}(D,V)},\;
\texttt{R\textsubscript{DW}(D,W)},\\
&\texttt{R\textsubscript{U}(U,B)},\;
\texttt{R\textsubscript{V}(V, A)}.
\end{aligned}
\]
All these relations are in \textbf{BCNF}:
\begin{itemize}
  \item Each \texttt{R\textsubscript{D*}} has determinant \texttt{D} which is a key in that relation.
  \item \texttt{R\textsubscript{U}(U,B)} and \texttt{R\textsubscript{V}(V,A)} have keys \texttt{U} and \texttt{V}, respectively.
  \item \texttt{R\textsubscript{CD}(C,D)} contains no nontrivial projected FD with a non-key LHS, hence BCNF.
\end{itemize}

\textbf{Losslessness.} Each BCNF split is lossless (decomposition by a dependency \texttt{X→Y} is lossless when the common part contains a key of one component). Since we only refined \texttt{R\textsubscript{CU}} by such a split, the overall join remains \textbf{lossless}.

\medskip
\textbf{Dependency preservation and comparison to (c).}
\begin{itemize}
  \item In (c), the 3NF design preserves all original dependencies, notably \texttt{CU → D} in \texttt{R\textsubscript{CU}}.
  \item In the BCNF design above, \texttt{CU → D} is \emph{not} preserved in any single relation (it can only be checked via a join), while the other dependencies (\texttt{D → U,A,E,V,W}, \texttt{U → B}, \texttt{V → A}) remain preserved in their respective relations.
\end{itemize}

\textbf{Answer.} The BCNF decomposition differs from (c) by replacing \texttt{R\textsubscript{CU}(C,U,D)} with \texttt{R\textsubscript{CD}(C,D)} and \texttt{R\textsubscript{DU}(D,U)}. It is \textbf{lossless} but not fully \textbf{dependency-preserving} (the FD \texttt{CU → D} is lost as a directly enforceable dependency). All resulting relations are in \textbf{BCNF}.
\end{enumerate}
\end{document}
