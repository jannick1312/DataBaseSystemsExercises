%% ================================================================
%% # DBIS Databases Hand-in Template
%% 
%% Template for students to hand-in their databases exercise solutions.
%% 
%% [Databases and Information Systems Group](https://dbis.dmi.unibas.ch/)
%%
%% ## Usage
%% 
%% Fill in the required fields and write your submission
%%
%% ## Issues
%%
%% See dbisdbhandin.sty for further information.
%% ================================================================
\documentclass{article}
\usepackage{dbisdbhandin}
\usepackage[english]{babel}
%% ================================================================
%%
%% General Information
%%
%% ================================================================
%%
%% Add your information here
\course       {Databases}
\semester     {Autumn 2025}
\exerciseno   {1}
\studenta     {Aiysha Frutiger}
\studentb     {Jannick Seper}
\studentc     {Luis Tritschler}
%% Comment if you do the exercises alone


%% ================================================================
%%
%% Common Packages
%%
%% ================================================================
%%
%% Useful common packages for this course

%% Include pdfs into latex
\usepackage{pdfpages}
%% \includepdf[pages=1,pagecommand={\pagestyle{fancy}}]{filename}

%% Drawing everything, with lots of libraries
\usepackage{tikz}
%% A library providing ER prefabs
\usetikzlibrary{er}
\usetikzlibrary{positioning,arrows.meta}
%% ================================================================
%%
%% Custom Packages
%%
%% ================================================================
%%
%% Add custom packages below:
%%

% \usepackage{mypackage}

%% ================================================================
%%
%% Custom Commands
%%
%% ================================================================
%%
%% DRY: Use commands when you use something often or you'd like
%% to define it only once
%%

\newcommand*{\TikZ}{Ti\textit{k}Z}


\begin{document}

\printfront

\task{}
\task{}
\begin{enumerate}[a)]
    \item \texttt{AHV $\rightarrow$ (FullName, Birthday, Zip, country)} \newline
        \texttt{FullName $\rightarrow$ (FirstName, LastName)} \newline
        \texttt{Birthday $\rightarrow$ YearOfBirth} \newline
        \texttt{(County, Zip) $\rightarrow$ City}

    The \texttt{AHV} uniquely identifies each resident. From these attributes, all others can be derived transitively:
    \texttt{FullName} determines \texttt{(FirstName, LastName)}, \texttt{Birthday} determines \texttt{YearOfBirth}, and \texttt{(Country, Zip)} determines \texttt{City}.
    We thought about including \texttt{(Country, City) $\rightarrow$ Zip} but there are cities with multiple zips (Zürich for example) and that's why we did not include this dependency.


    \item With this functional dependencies we compute the attribute closure for \texttt{AHV}. \texttt{F$^+$} is \texttt{(AHV, FullName, Birthday, Zip, Country, FirstName, LastName, YearOfBirth, City) \newline = sch(Resident)}. Since no subset of \texttt{AHV} determines all attributes, \texttt{AHV} is minimal and therefore the only candidate key.
    

    \item The relation Resident is in 2NF, since the only candidate key is AHV and therefore no partial dependencies on a subset of a composite key can exist. It is not in Third Normal Form, because there are several transitive dependencies. For example: \texttt{AHV $\rightarrow$ (Zip, country)} and \texttt{(County, Zip) $\rightarrow$ City}, hence City is transitively dependent on AHV.
    Version in 3NF: \newline
    \texttt{Resident(\uline{AHV}, \uwave{FullName}, \uwave{Birthday}, \uwave{Zip, Country})} \newline
    \texttt{Name(\uline{FullName}, FirstName, LastName)} \newline
    \texttt{Birthday(\uline{Birthday}, YearOfBirth)} \newline
    \texttt{Location(\uline{Zip, Country}, City)} 
\end{enumerate}

\task{}

\task{}

\task{}

\task{}

\end{document}
