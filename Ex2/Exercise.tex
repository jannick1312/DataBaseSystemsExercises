%% ================================================================
%% # DBIS Databases Hand-in Template
%% 
%% Template for students to hand-in their databases exercise solutions.
%% 
%% [Databases and Information Systems Group](https://dbis.dmi.unibas.ch/)
%%
%% ## Usage
%% 
%% Fill in the required fields and write your submission
%%
%% ## Issues
%%
%% See dbisdbhandin.sty for further information.
%% ================================================================
\documentclass{article}
\usepackage{dbisdbhandin}
\usepackage[english]{babel}
%% ================================================================
%%
%% General Information
%%
%% ================================================================
%%
%% Add your information here
\course       {Databases}
\semester     {Autumn 2025}
\exerciseno   {1}
\studenta     {Aiysha Frutiger}
\studentb     {Jannick Seper}
\studentc     {Luis Tritschler}
%% Comment if you do the exercises alone


%% ================================================================
%%
%% Common Packages
%%
%% ================================================================
%%
%% Useful common packages for this course

%% Include pdfs into latex
\usepackage{pdfpages}
%% \includepdf[pages=1,pagecommand={\pagestyle{fancy}}]{filename}

%% Drawing everything, with lots of libraries
\usepackage{tikz}
%% A library providing ER prefabs
\usetikzlibrary{er}
\usetikzlibrary{positioning,arrows.meta}
%% ================================================================
%%
%% Custom Packages
%%
%% ================================================================
%%
%% Add custom packages below:
%%

% \usepackage{mypackage}

%% ================================================================
%%
%% Custom Commands
%%
%% ================================================================
%%
%% DRY: Use commands when you use something often or you'd like
%% to define it only once
%%

\newcommand*{\TikZ}{Ti\textit{k}Z}


\begin{document}

\printfront

\task{}
Given excerpt of the relational schema plus added relations and integrety constrain befor the first bullet point.
\lstinputlisting[language=SQL,frame=single,numbers=left,numberstyle=\tiny]{task1_givennew.sql}
\newpage
\begin{itemize}
    \item 
    \textit{The title of the lecture has to be unique and may not be altered if any exercise is
    available for the lecture.}\\
    The first part of the point is already enforced bc the title of the lecture is a \texttt{PRIMARY KEY} and therefore must be unique. The second part can be enforced with this addition.
    \lstinputlisting[language=SQL,frame=single,numbers=left,numberstyle=\tiny]{task1_1.sql}
    \item 
    \textit{For a lecture, no more than 10 credit points may be awarded.}\\
    This part can be enforced with this assertion.
    \lstinputlisting[language=SQL,frame=single,numbers=left,numberstyle=\tiny]{task1_2.sql}
    \item 
    \textit{Lecturers may give multiple lectures.}\\
    This part is beeing enforced with this.
    \lstinputlisting[language=SQL,frame=single,numbers=left,numberstyle=\tiny]{task1_3.sql}
    \item 
    \textit{A lecture may include several exercises. An exercise always belongs to exactly one
    lecture.}\\
    The first part is already enforced via the foreign key while for the second part we have to add the \texttt{NOT NULL} so we guaratee that e exersise must be in one lecture.
    \lstinputlisting[language=SQL,frame=single,numbers=left,numberstyle=\tiny]{task1_4.sql}
    \item 
    \textit{Before a new author is entered into the system, it should be checked that no other
    author with the same first name, last name and title is present.}\\
    This part is beeing enforced with this.
    \lstinputlisting[language=SQL,frame=single,numbers=left,numberstyle=\tiny]{task1_5.sql}

\end{itemize}


\task{}

\task{}

\task{}
NOT sure yeeeet.

\end{document}
